\section{Value of Uninformed Orderflow for LPs} \label{section:lp-oflow-value}
We proceed with the following methodology to determine the revenue that uninformed orderflow creates for LPs in an Uniswap V3 pool.
First, we use historical order data to filter orders into informed vs uninformed orders, and we estimate the distribution of sizes of uninformed orders.
Next, we run a number of simulations in which we sample from this estimated distribution of uninformed order sizes and compute the amount of revenue generated to LPs from that uninformed order; this revenue quantity includes the revenue that would be generated by sandwiching the uninformed order. This allows us to both determine the value created by uninformed orders of various sizes and determine the mean value per dollar of uninformed orderflow.
We then use these values to estimate the influx of LP positions that we would expect to see as a result of an increase in LP revenue. To do this, we use historical pool data to filter positions into \textit{active} and \textit{passive}, based on how long they have been open, and we assume that the active positions will behave more rationally with respect to increases in LP revenues, whereas we assume that passive positions will be less responsive. By doing this, we can place a conservative estimate on the influx of new LP positions that would result from an increase in LP revenues.

\subsection{The distribution of uninformed order sizes.}
% 1. We sample from historical Uniswap trades to find the distribution of uninformed orderflow sizes. We outline our methodology for determining which orders are informed in the appendix.
We estimate the distribution of uninformed order sizes by filtering historical orders into informed and uninformed orders, then approximating the distribution of historical trade sizes. To perform this filtering, we


\subsection{Simulating uninformed orderflow's impact on LP revenue.}
% 2. We sample from these historical trades to find the liquidity present at the time those swaps are made. We can use this to fit a function from swap size -> liquidity, although that regression may not be necessary / have statistical significance.

% 3. We run simulations whereby we sample a swap size, compute the amount of fees that it generates for the LPs, and find the amount of capital vs dollar of swap size.


\subsection{Relating LP revenue to LP positions.}
% 4. We then estimate the amount of additional liquidity we would expect in the pool, due to this increase in revenues. We do this in a rather unsophisticated way, by computing (new positions) = (new revenue)/(old revenue) * (old positions), and from this we could compute some liquidity metric L(new positions) and compare it to a liquidity metric L(old positions). However, this is almost certainly an over-estimation of the new liquidity, since it is unlikely that all of the original positions are linearly related to protocol revenues. 
% 5. We could do something to estimate which capital is _sticky_ vs _not sticky_ by instead just looking at (old positions | position has existed over a shorter term). These actively-managed positions are more likely to be elastic to LP revenues, and presumably they are managed by "smart money" market makers who would increase their capital deployed if there was an opportunity to earn more revenue. This does add another parameter to our problem -- the cutoff for how long a position has existed -- but we can see how it varies and adjust to something that's reasonable. We'll provide a couple different params. We'll choose an aggressively short term filter for the positions, then compute (new active positions) = (new revenue)/(old revenue) * (old positions), then compute (new positions) = (old positions - old active positions + new active positions), then compute L(new positions). This is the most practical approach we could derive for estimating the increase in LPs. Assumptions. (1) Uniswap uninformed swap size distribution will be the same in the future as it is today, (2) this liquidity model is sufficient to capture liquidity, (3) more like a possible issue is that we're not measuring gains to passive liquidity, (4) ?
% Another model for computing additional liquidity is to compute the PnL of LPing, see how this increases (in expectation) with more fee revenue, then see how you'd Markowitz-style optimize it, then see what the new liquidity positions would be. This would require an immense amount of assumptions though: uniform LP risk-reward preferences, uniform LP strategies, usage of a risk-free rate, etc.