\section{Future Work} \label{section:future-work}

\textbf{Sandwich Attack Data Analysis}. We queried 6 months of Uniswap swaps for a number of top Uniswap pools. When we started analyzing the sandwich data, we were surprised by the magnitude and informational properties of sandwich attack. It would be great to see more work in the following areas: finding the source of the sandwich volume, explaining the massive variance in sandwich attack sizes, and exploring sandwich attack mitigations at the protocol or interface level.

\textbf{Optimal Interface Orderflow Procurement}. When procuring orderflow from interfaces, how much should Uniswap be willing to pay? One approach, which we provide in our paper, is for Uniswap to state its price on a per-interface basis, then negotiate with each interface; this is akin to running multiple posted-price auctions. Instead, Uniswap could procure interface orderflow by publishing a request for orderflow, then allowing interfaces to bid in advance to determine how much Uniswap protocol would pay. Utilizing a multi-unit Dutch-style auction may allow Uniswap to procure this orderflow at a deep discount, based on interfaces' willingness to route at lower prices. % For instance, this request for orderflow could be a request for \$10M of volume on USDC-ETH-0.05\% over the next week, and we could orchestrate a 

\textbf{Alternative Revenue Streams}. The protocol fee is the canonical built-in revenue generation mechanism, but are there other ways that the protocol could generate revenue? One idea, akin to Skip protocol's Osmosis solution, is for the Uniswap Foundation to sponsor the creation of a block builder that back-runs users' transactions and give the proceeds to the protocol's treasury; see also A$^2$MM for a similar idea \cite{zhou2021a2mm}.

\textbf{Behavioral Economics of Retail DeFi Users}. Our analysis was based in financial economics and on-chain data, but we do not analyze the actual people creating uninformed orders. What consumer incentive programs have worked -- and not worked -- historically? Can we apply those learnings to Uniswap? Also, what is the profile of an Uniswap user who generates uninformed orderflow, and how can we find more customers who fit their same customer profile?

\textbf{The Orderflow-Liquidity Feedback Loop}. Feedback loops are interesting, as they can tend toward exponential growth or decay. We briefly outline a feedback loop between orderflow and liquidity in appendix \ref{appendix:oflow-liq-feedback}, but there is still ample greenfield work to be done on this topic. What is the relationship between uninformed orderflow and liquidity? The relationship between liquidity and DEX aggregator volume? Can Uniswap dominate DEX aggregator volume by dominating an orderflow-liquidity feedback loop, and if so, is there a low-cost strategic move that would catalyze this?
