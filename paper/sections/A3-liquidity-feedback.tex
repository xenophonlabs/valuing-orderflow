\section{The Orderflow-Liquidity Feedback Loop} \label{appendix:oflow-liq-feedback}
    The model we provide for valuing orderflow in the paper is meant to give a lower bound on the orderflow value, and we do so by looking only at the amount of value created nearly immediately following an uninformed order, where liquidity is held constant. While we assume that this model makes sense for small additions of orderflow, it is clear that if a sufficiently large amount of uninformed orderflow comes in, then liquidity will increase. Here, we present a rough model for how one might model these interaction effects between uninformed orderflow and liquidity. This section is intended for nerd-sniping purposes only, and should be read as a first-pass on the interaction effects betwee uninformed orderflow and liquidity; we do \textit{not} endorse this methodology.

    \subsection{The model}
    Suppose there are types of of orderflow that go to Uniswap pools: arbitrage volume, DEX aggregator volume, and retail volume. Let $t$ denote time, and initially $t=0$. Let $L_t$ be a metric for liquidity at time $t$; we do not define what this metric should be. Let $v_{arb,t}$ be the rate of arbitrage volume at time $t$; $v_{agg,t}$ be the rate of DEX aggregator volume at time $t$; let $v_{ret,t}$ be the rate of retail volume at time $t$. 
    Assume the following relationships hold:
        \begin{align*}
            \frac{\partial L_t}{\partial v_{arb,t}} & = 0, \\
            \frac{\partial L_t}{\partial v_{agg,t}} & = \beta_{agg}, \text{ and} \\
            \frac{\partial L_t}{\partial v_{ret,t}} & = \beta_{ret},
        \end{align*}
    for some $\beta_{agg}, \beta_{ret} \in \mathbb R^+$.

    These $\beta_{*}$ values should be interpreted as the elasticities of liquidity, with respect to the different types of volume. Implicit here is that an increase in arbitrage volume will not cause an increase in liquidity. Assume that all of the liquidity is explained by the volume, such that
        \begin{equation} \label{eq:uni-liquidity}
            L_t = L_{agg,t} + L_{ret,t} = \beta_{agg} v_{agg,t} + \beta_{ret} v_{ret,t},
        \end{equation}

    where $L_{agg,t}$ is the liquidity due to aggregator volume, and $L_{ret,t}$ is the liquidity due to retail volume.

    We assume that there exist $L_{rest,t}$ units of liquidity across other DEX aggregator-reachable venues; we also assume that the the total rate of volume coming from a DEX aggregator at time $t$ is $V_{agg,t}$. Now we make what is perhaps our most egregious assumption, that the rate of DEX aggregator volume given to the Uniswap pool is proportional to the Uniswap pool's liquidity share:
        \begin{equation} \label{eq:agg-volume}
            v_{agg,t} = \frac{L_t}{L_{rest, t}+L_t}V_{agg,t}.
        \end{equation}

    Combining equations \ref{eq:uni-liquidity} and \ref{eq:agg-volume}, we get that if retail volume changes between time $0$ and time $t$, and causes a change in liquidity, but the rest of the DEXs' liquidity stays the same, then the rate of aggregator volume can be represented as the following:
        \begin{align} \label{eq:v-agg-wrt-retail}
            \begin{split}
                v_{agg,t} 
                & = \frac{(-\beta_{agg} + L_{ret,t}) - \sqrt{(\beta_{agg}-L_{ret,t})^2+4\beta_{agg}(V_{agg,0}L_{ret,t}-L_{rest,0})}}{-2\beta_{agg}} \\
                & = \frac{(-\beta_{agg} + \beta_{ret} v_{ret,t}) - \sqrt{(\beta_{agg}-\beta_{ret} v_{ret,t})^2+4\beta_{agg}(V_{agg,0}\beta_{ret} v_{ret,t}-L_{rest,0})}}{-2\beta_{agg}}.
            \end{split}
        \end{align}

    Given the changes in aggregator volume, we get that the new arbitrage volume is
        \begin{align}
            \begin{split}
                v_{arb,t} 
                & = \omega L_t \\
                & = \omega \left( \beta_{agg} v_{agg,t} + \beta_{ret} v_{ret,t} \right).% \\
                % & = \omega \frac{(-\beta_{agg} + \beta_{ret} v_{ret,t}) - \sqrt{(\beta_{agg}-\beta_{ret} v_{ret,t})^2+4\beta_{agg}(V_{agg,0}\beta_{ret} v_{ret,t}-L_{rest,0})}}{-2} \\
                % & \ \ \ + \omega \beta_{ret} v_{ret,t}.
            \end{split}
        \end{align}
    % TODO check math on this, doesn't seem to be correct in the desmos

    We can then calculate the quantity of fees generated by the protocol as follows:
        \begin{align}
            F_t & = \gamma \phi (v_{arb,t} + v_{agg,t} + v_{ret,t}),
        \end{align}
    where $F_t$ is the rate of protocol fees per unit time.

    We could then use this fee function to calculate the change in fees, $\Delta F_t$, that we would expect given a change in retail trades, $\Delta v_{ret,t}$. This $\Delta F_t / \Delta v_{ret,t}$ quantity would account for the interaction effects between arbitrageurs, DEX aggregators, and retail traders; by comparison, the lower bound we establish in the paper ignores these interaction effects.

    \subsection{Discussion}

    \textbf{Issues}
    Although this model of the interaction effects is interesting, there is no indication that it is correct. We ascribe continuous functions everywhere: a continuous relationship between liquidity and DEX aggregator volume rate, a constant increase in liquidity per unit of aggregator volume rate, an constant increase in liquidity per unit of retail volume rate, etc. These models are clearly false in real life; all of these codependencies between liquidity and volume must be implemented by economic agents, and we have no indication that this is a realistic assumption. Furthermore, even if this model were sufficient, it would be quite challenging to measure the $\beta_*$ and $\omega$ parameters.

    \textbf{Learnings}
    Despite its issues, we believe this model provides a helpful mental framework for assessing the interactions between orderflow and liquidity. For instance, we would not be surprised if the relationship between aggregator volume rate and liquidity followed a similar function to that provided in equation \ref{eq:agg-volume} for large changes in liquidity $L_t$. But this is speculation.

    Interestingly, if this model's assumption that liquidity is not responsive to arb volume, that retail volume is not responsive to liquidity, and that DEX aggregators' rate of volume is fixed, is correct, then it would seem that this is a negative feedback loop that reaches equilibrium at a finite value of DEX aggregator and arbitrage volume. Of course, this should be trivial, given the fact that arbitrage volume is the only infinitely scalable volume term in this model, and we assume that it does not lead to increases in liquidity. In fact, we hypothesize that this would be a feedback loop under much weaker conditions, for any setup where the (increase in retail volume) $\rightarrow$ (increase in DEX liquidity) $\rightarrow$ (increase in retail volume) loop has a finite sum over all increases in retail volume. Weakening our model's assumptions would be an excellent further area of research on this problem.        
    
    
    % Assume that changes in $v_{arb,t}$have no effect on liquidity; assume that increases in arbitrage, in isolation, have a

    
    % Assume that Uniswap liquidity has elasticity $\beta_{agg}$ to DEX aggregator volume, 