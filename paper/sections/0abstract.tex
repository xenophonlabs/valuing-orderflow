\begin{abstract}
    % The Uniswap protocol is composed of automated market maker where passive liquidity providers (LPs) earn fees proportional to the volume they fill.
    % % In order for the protocol's liquidity providers to remain on the protocol in the long run, they must earn enough fees to offset their losses. % TODO clean up the language here
    % Here, we compare the revenue that the impact that uninformed taker volume has on revenues for the protocol's liquidity providers, and conversely how much marginal liquidity we would expect as a result of increasing uninformed taker volume.
    % The Uniswap Protocol requires both market liquidity and price discovery, and it satisfies these requirements via liquidity providers (LPs) and arbitrageurs, respectively. LPs are incentivized by earning a fee proportional to the volume that they fill. 
    % It is existentially necessary to the Uniswap Protocol's long-term viability that its liquidity providers (LPs) make profit in expectation.
    % Uninformed orderflow, i.e. trades that are not predictive of future price movements, has a diffe
    In this paper, we analyze how \textit{uninformed orderflow} -- orders that are not predictive of future price movements -- affect the revenue earned by liquidity providers on the Uniswap Protocol.
    We also present an opinionated framework for how much the protocol should value liquidity.
    Finally, we conclude by providing a recommended upper bound on how much the protocol should pay for uninformed orderflow.
\end{abstract}